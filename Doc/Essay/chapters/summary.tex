\chapter{Zusammenfassung}
\label{chap:Zusammenfassung}

\chapterstart

Als Projekt wurde ein WebServices für Studierende der FH-Joanneum entwickelt. Dieses Service wurde in zwei Sprachen, JavaScript und Swift, implementiert. Für das entwicklen des JavaScript-Servers wurde Node.js mit dem Framework Express.js verwendet. Node.js ist eine leichte und effiziente Laufzeitumgebung für JavaScript. Für den Server selbst, wurde das Framework Express.js verwendet. Das Service wurde im Zug der Lehrveranstaltung "Rich Internet Applikation" von Michael Roitinger, Stefan Moder und Christian Hofer entwickelt. Der Server musste für die Lehrveranstalltung verschiedene Kriterien einhalten, wie das dynamische Erstellen von Webite Inhalten, das verwalten von Session, MVC-Pattern und einige weitere. Diese Kriterien dienten für den später entwickelten Swift-Server als Reverenz.\\
Um den Swift-Server zu entwickeln, musste zuerste eine Linux Distribution gefunden werden, auf der Swift kompatibel war. Aufgrund mehrerer Faktoren und schlussendlich auch aufgrund dessen, dass die Swift-Community Ubuntu empfahl, wurde für die Entwicklung Ubuntu 16.04 LTS als virtuelle Maschine im Oracels "VirtualBox" aufgesetz. Nach der Installation von Swift auf der Ubuntu Distribution, wurde der Server mit dem Swift Framework Perfect erstellt. Aus den Webframeworks Perfect, Vapor, Kitura, Zewo und Swifter wurde das Perfect Framework für diese Arbeit gewählt. Nachdem Abwegen der Vorzüge der einzelnen Frameworks, kristallisierte sich das Perfect Framework, unteranderem durch die große Menge an zusätzlichen Werkzeugen, eine detaillierte und gut dokumentierte API, gründlichen FAQs, jeder Menge Beispielcode und einer Community die auf die Fragen eingeht, heraus. Die Verwendung eines Frameworks, ist notwendig, um einen Bezug für Express.js herzustellen. Unter Verwendung von Swift gegeben Bibliotheken und den Erweiterungen von Express wurde der Swift-Server entwickelt und hinsichtlich Code, inklusive der Funktionalität von Perfect und der Abarbeitung von Request mit dem JavaScript-Server verglichen. \\
Der Leistungsvergleich erfolgte ebenfalls auf der Ubuntu Distribution. Davor wurden die Server mit YSlow verglichen und auf den gleichen Stand gebracht, um sicherzustellen, das der Vergleich auch aussagekräftig ist. Der Vergleich selbst wurde mit dem Tool "Apache Bench" durchgeführt. Dabei wurden beide Server mit 5.000, 10.000, 20.000, 30.000, 40.000 und 50.000 Requests beauftragt, wobei jeweils 500 Request zeitgleich erfolgten. Die damit erreichten Benchmarks wurden ausgewertet und ein Fazit gezogen.\\
Der Vergleich hat gezeigt, dass die Server sich sehr ähnlich sind, jedoch der Node.js Server die Requests effizienter abarbeitet. Dies ist darauf zurückzuführen, dass der Swift Server bei den I/O-Operationen blockiert und derweil keine nachkommenden Requests behandelt. Außerdem waren für den Swift-Server ein Workaround für die Templatefunktionalität, die mit Handlebars erreicht wurde, notwendig. Dieses Workaround besitzt keine Callback Funktionen wie die von Handlebars, wobie dies bei Perfect anscheindend auch nicht vorgesehen ist. Auch wenn Perfect eine stärkere Kompression aufweist, dass bei der Übermittlung vorteilhafter ist als bei Express und dem Modul Compression, benötigt es einen höheren Rechenaufwand, wodurch der Server wieder langsamer wird.

 \chapterend